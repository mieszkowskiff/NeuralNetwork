\documentclass{article}
\usepackage{graphicx} % Required for inserting images

\title{Sieci Neuronowe, projekt II - Sieci Hopfielda}
\author{Filip Mieszkowski, Stanisław Kurzątkowski}
\date{November 2024}
\usepackage[T1]{fontenc}
\begin{document}

\maketitle

\section{Wstęp}
Celem tego projektu było zaimplementowanie sieci Hopfielda, której zadaniem jest rozpoznawanie wzorców. 
Sieć Hopfielda jest jedną z najstarszych sieci neuronowych, została zaproponowana przez Johna Hopfielda w 1982 roku. 

\subsection{Budowa sieci}
Sieć składa się z $n$ neuronów, każdy z tych neuronów jest połączony z każdym innym neuonem wagą. 
Stan każdego neuronu może przyjmować tylko dwie wartości, -1 lub 1.
Wagę pomiędzy neuronami $i$ i $j$ będziemy oznaczali jako $w_{ij}$.
Wagi te są symetryczne, co oznacza, że $w_{ij} = w_{ji}$. 
Dodatkowo, waga $w_{ii}$ jest równa 0, co oznacza, że neuron nie jest połączony z samym sobą.
Wagi sieci będziemy przechowywać w macierzy wag $W$, gdzie $W = [w_{ij}]$.
Z wymienionych powyżej własności wynika, że macierz wag jest symetryczna oraz posiada na przekątnej same zera.


\subsection{Cel sieci}
Sieć Hopfielda jest siecią asocjacyjną, co oznacza, że jest w stanie zapamiętać wiele wzorców i potrafi je rozpoznać, 
nawet jeśli są one zaszumione. Wzorce te są przechowywane w macierzy wag, która jest obliczana na podstawie wzorców uczących.


\section{}
 
\end{document}
