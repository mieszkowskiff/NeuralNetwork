\documentclass{article}
\usepackage{graphicx} % Required for inserting images


\title{Sieci Neuronowe, projekt II - Sieci Hopfielda}
\author{Filip Mieszkowski, Stanisław Kurzątkowski}
\date{Listopad 2024}
\usepackage[T1]{fontenc}
\begin{document}

\maketitle

\section{Wstęp}
Celem tego projektu jest zaimplementowanie sieci Hopfielda, której zadaniem jest rozpoznawanie wzorców. 
Sieć Hopfielda jest jedną z najstarszych sieci neuronowych, została zaproponowana przez Johna Hopfielda w 1982 roku. 

\subsection{Budowa sieci}
Sieć składa się z $n$ neuronów, każdy z tych neuronów jest połączony z każdym innym neuonem wagą. 
Stan każdego neuronu może przyjmować tylko dwie wartości, -1 lub 1.
Wagę pomiędzy neuronami $i$ i $j$ będziemy oznaczali jako $w_{ij}$.
Wagi te są symetryczne, co oznacza, że $w_{ij} = w_{ji}$. 
Dodatkowo, waga $w_{ii}$ jest równa 0, co oznacza, że neuron nie jest połączony z samym sobą.
Wagi sieci będziemy przechowywać w macierzy wag $W$, gdzie $W = [w_{ij}]$.
Z wymienionych powyżej własności wynika, że macierz wag jest symetryczna oraz posiada na przekątnej same zera.


\subsection{Cel sieci}
Sieć Hopfielda jest siecią asocjacyjną, co oznacza, że jest w stanie zapamiętać wiele wzorców i potrafi je rozpoznać, 
nawet jeśli są one zaszumione. Wzorce te są przechowywane w macierzy wag, która jest obliczana na podstawie wzorców uczących.


\section{Działanie sieci}
Działanie sieci Hopfielda można podzielić na dwa etapy: uczenie oraz rozpoznawanie wzorców.

\subsection{Rozpoznawanie wzorców}
Rozpoznawanie wzorców polega na prezentowaniu sieci wzorca, a następnie obserwacji zmian w stanie neuronów.
Załóżmy, że wagi pomiędzy neuronami są ustalone.
Pokazujemy sieci wzorzec, który chcemy rozpoznać, tzn ustalamy wartości neuronów na wartości z wzorca.
Następnie, obserwujemy zmiany w stanie neuronów, aż do momentu, gdy sieć osiągnie stan stabilny.
Dzieje się to w następujący sposób: wybieramy $i$-ty neuron, 
a następnie obliczamy sumę ważoną wejść do tego neuronu, czyli:
$$h_i = \sum_{j=1}^{n} w_{ij} x_j$$
gdzie $x_j$ to stan $j$-tego neuronu (ponownie przyjmujemy $w_{ii} = 0$). 
W zależności od wartości $h_i$ neuron przyjmuje wartość 1 lub -1, 
tzn jeżeli $h_i > b_{i}$ to $x_i = 1$, w przeciwnym wypadku $x_i = -1$,
gdzie $b_{i}$ to próg aktywacji neuronu $i$. Proces ten powtarzamy dla wszystkich neuronów, aż do momentu, gdy stan sieci się ustabilizuje.
Możemy tutaj skorzystać z dwóch rozwiązań: asynchronicznego oraz synchronicznego.
W rozwiązaniu asynchronicznym przeprowadzamy opsiany wyżej proces dla każdego neuronu osobno,
natomiast w rozwiązaniu synchronicznym dla wszystkich neuronów naraz. W przypadku 
rozwiązania asynchronicznego neurony mają z góry ustaloną kolejść, w której są aktualizowane.
Kolejność ta nie zmienia się w kolejnych iteracjach.


\subsection{Uczenie Hebbowskie}
Uczenie sieci Hopfielda polega na obliczeniu macierzy wag $W$ oraz progów aktywacji $B = [b_i]$ 
na podstawie wzorców uczących. Załóżmy, że mamy $p$ wzorców uczących, które będziemy oznaczać jako $X = [x^{(1)}, x^{(2)}, \ldots, x^{(p)}]$.
Wówczas wagi ustalamy zgodnie ze wzorem:
$$w_{ij} = \frac{1}{p} \sum_{\mu=1}^{p} x_i^{(\mu)} x_j^{(\mu)}$$
zaś progi aktywacji ustalamy zgodnie ze wzorem:
$$b_i = \frac{1}{p}\sum_{\mu=1}^{p} x_i^{(\mu)}$$
gdzie $x_i^{(\mu)}$ oznacza $i$-ty neuron $mu$-tego wzorca uczącego.
Warto zauważyć, że wagi są symetryczne, a progi aktywacji są ustalane na podstawie średniej wartości neuronów w danym wzorcu.

\subsection{Dlaczego uczenie Hebbowskie działa?}
Zdefiniujmy funkcję energii sieci Hopfielda jako:
$$E = -\frac{1}{2} \sum_{i=1}^{n} \sum_{j=1}^{n} w_{ij} x_i x_j + \sum_{i=1}^{n} b_i x_i$$
gdzie $x_i$ to stan $i$-tego neuronu.
Możemy zauważyć, że funkcja ta jest zdefiniowana w taki sposób, że dla każdego wzorca uczącego $x^{(\mu)}$ jest minimum lokalne.
Dodatkowo zauważmy, że pochodna funkcji energii po $x_i$ jest równa:
$$\frac{\partial E}{\partial x_i} = -\sum_{j=1}^{n} w_{ij} x_j + b_i$$
Zatem zmieniając wartość neuronu $i$-tego w kierunku $-\frac{\partial E}{\partial x_i}$ powinna zmniejszać się energia sieci.
Nie zawsze musi się tak dziać, bowiem zmiana nawet w dobrym kierunku nie następuje w sposób ciągły, a od razu o dwie jednostki,
co może okazać się "zbyt dużym krokiem". Oczekujemy jednak, że po wielu iteracjach energia sieci będzie maleć,
co w końcu doprowadzi do osiągnięcia stanu stabilnego, który jest minimum lokalnym funkcji energii.

\subsection{Fałszywe minima lokalne: uczenie Oja}
Załóżmy, że sieć Hopfielda dotrze po pewnym czasie do minimum funkcji energii.
Niestety nie mamy żadnej pewności, że jest to minimum związane z jednym z wzorców uczących.
Jeżeli wzorców uczących było więcej niż 1, to oprócz minimów związanych ze wzorcami uczącymi 
mogą istnieć również inne minima lokalne, które nie są związane z żadnym z wzorców uczących.
W takim przypadku sieć może nie rozpoznać żadnego z wzorców uczących, a zamiast tego zbiegnie do jednego z fałszywych minimów lokalnych.
Aby temu zapobiec, można skorzystać z uczenia Oja, które polega na dodaniu do reguły uczenia Hebbowskiego czynnika korygującego.
Po zastosowaniu uczenia Hebbowskiego, wagi sieci są modyfikowane zgodnie ze wzorem:
$$\Delta w_{ij} = \frac{\eta}{n} \sum_{s=1}^{n}y_i^{(s)}(x_j^{(s)} - y_i^{(s)}w_{ij})$$
gdzie $y_i = \sum_{j = 1}^{n} w_{ij}x_j$. Oczywiście nie aktualizujemy wagi $w_{ii}$,
gdyż chcemy, żeby została ona równa $0$.
Taka korekta jest dokonywana kilka razy, dla pewnego ustalonego $\eta$.
W praktyce pomaga to uniknąć fałszywych minimów lokalnych, co zwiększa skuteczność sieci.

\section{Eksperymenty}
W ramach projektu przeprowadziliśmy kilka eksperymentów, które miały na celu zbadanie skuteczności sieci Hopfielda.
Kod źródłowy eksperymentów $1$ oraz $2$ znajduje się w pliku $eksperyment.py$.
W zależności od dobranego ziarna losowości, wyniki mogą się nieznacznie różnić.

\subsection{Eksperyment 1: rozpoznawanie wzorców zaszumionych}\label{eksperyment1}
W pierwszym eksperymencie sprawdziliśmy, jak uczenie Oja wpływa na skuteczność sieci.
W tym celu stworzyliśmy sieć Hopfielda, która miała za zadanie rozpoznawać wzorce zaszumione.
Sieć o $625$ neuronach została wyszkolona na $6$ wzorcach, a następnie testowana na wzorcach zaszumionych.
Zaszumienie polegało na zmianie $10\%$ losowo wybranych pikseli na przeciwną wartość.
Zbiorem uczącym był zbiór $large-25x25$. Zwykłe uczenie Hebbowskie nie pozwalało na rozpoznanie wsztystkich wzorców zaszumionych,
dwa zbiegły do minimów lokalnych, które nie były związane z żadnym z wzorców uczących.
Poniżej prezentujemy te dwa źle rozpoznane wzorce uczące:
\begin{center}
    \includegraphics[width=0.7\textwidth]{./../../data/hopfield/eksperyment12/train/p2.png}
    \includegraphics[width=0.7\textwidth]{./../../data/hopfield/eksperyment12/train/p4.png}
\end{center}
Po zaszumieniu wzorce te wyglądały następująco:
\begin{center}
    \includegraphics[width=0.7\textwidth]{./../../data/hopfield/eksperyment12/noise/n2.png}
    \includegraphics[width=0.7\textwidth]{./../../data/hopfield/eksperyment12/noise/n4.png}
\end{center}
Po zaprezentowaniu sieci zaszumionych wzorców, sieć nie była w stanie ich rozpoznać, zbiegła
do następujących minimów lokalnych:
\begin{center}
    \includegraphics[width=0.7\textwidth]{./../../data/hopfield/eksperyment12/hebb/h2.png}
    \includegraphics[width=0.7\textwidth]{./../../data/hopfield/eksperyment12/hebb/h4.png}
\end{center}
Szczególnie przy pierwszym z tych minimów lokalnych, można zauważyć, że jest on swgo rodzaju "połączeniem" 
obydwu wzorców uczących.
\subsection{Eksperyment 2: rozpoznawanie wzorców zaszumionych z uczeniem Oja}
W tym eksperymnecie będziemy rozpoznawali dokładnie te same wzorce co w poprzednim eksperymencie, 
ale z użyciem uczenia Oja. Ustalmy liczbę iteracji uczenia Oja na $30$. Zaś $\eta = 0.00001$.
Jest to bardzo mała wartość, jednakże biorąc większą wartość $\eta$ proces uczenia szybko rozbiegał do nieskończoności.
Po zastosowaniu uczenia Oja, sieć była w stanie rozpoznać wszystkie wzorce zaszumione.
Co również interesujące, po zastosowaniu uczenia Oja sieć zbiegała do do minimów lokalnych już w pierwszej iteracji
(czyli po jednokrotnym zaktualizowaniu każdego neuronu), zaś w przypadku zwykłego uczenia Hebbowskiego wzorzec, który
zbiegł do niepoprawnego minimum lokalnego zbiegł do niego po trzech iteracjach.

\subsection{Eksperyment 3: rozpoznawanie liter alfabetu}
W tym eksperymencie sprawdziliśmy, jak sieć Hopfielda radzi sobie z rozpoznawaniem liter alfabetu.
W tym celu skorzystaliśmy z zbioru uczącego $letters-14x20$, który zawiera $26$ liter alfabetu.
Sieć została wyszkolona na $26$ literach, a następnie testowana na literach zaszumionych.
Zaszumienie polegało na zmianie $10\%$ losowo wybranych pikseli na przeciwną wartość.
Minimum lokalne, do którego zbiegła sieć dla wszystkich zaszumionych liter w przypadku uczenia 
Hebbowskiego wyglądało następująco:
\begin{center}
    \includegraphics[width=0.7\textwidth]{./../../data/hopfield/eksperyment3/hebb/h1.png}
\end{center}
Z kolei w przypadku uczenia Oja, sieć zbiegła do dwóch następujących minimów lokalnych:
\begin{center}
    \includegraphics[width=0.7\textwidth]{./../../data/hopfield/eksperyment3/oja/o1.png}
    \includegraphics[width=0.7\textwidth]{./../../data/hopfield/eksperyment3/oja/o23.png}
\end{center}
Jak widać, zarówno uczenie Hebbowskie jak i Oja nie pozwoliło na rozpoznanie liter zaszumionych.
Prawdopodobnie wynika to z faktu, że obrazki liter alfabetu są ze sobą silnie skorelowane, 
szczególnie, że w podanym zbiorze testowym wszystkie one znajdują się w lewym górnym rogu obrazka.
Korelacja wzorców uczących stanowi dla sieci Hopfielda poważny problem, o czym można przeczytać na przykład
w \cite{ref1}.

\subsection{Eksperyment 4: porównanie podobnych zbiorów uczących}
W tym eksperymencie uczyliśmy sieć Hopfielda na zbiorze $large-25x25$ oraz $large-25x50$.
Są to niemalże te same wzorce, z tą różnicą, że w zbiorze $large-25x50$ dolna połowa każdego obrazka jest
cała biała. Wyniki uczenia na zbiorze $large-25x25$ opisaliśmy w rozdziale \ref{eksperyment1}.
W przypadku uczenia na zbiorze $large-25x50$, sieć uczona metodą Hebbowską zupełnie soobie nie poradziła
i zbiegła do minimów lokalnych, które nie były związane z żadnym z wzorców uczących, zaś wyglądały następująco:
\begin{center}
    \includegraphics[width=0.7\textwidth]{./../../data/hopfield/eksperyment4/hebb/h1.png}
    \includegraphics[width=0.7\textwidth]{./../../data/hopfield/eksperyment4/hebb/h5.png}
\end{center}
Uczenie Oja dla odmiany pozwoliło na rozpoznanie wszystkich wzorców zaszumionych.
Liczba iteracji równa była $30$, zaś $\eta = 0.000002$ (zauważmy, że $\eta$ jest mniejsza, niż w przypadku 
zbioru $large-25x25$).

\subsubsection{Wytłumaczenie wyników}
Wyniki eksperymentów pokazują, że uczenie Oja jest znacznie bardziej skuteczne niż uczenie Hebbowskie.
Rozważmy neuron należący do górnej części obrazka. W przypadku uczenia Hebbowskiego, neuron ten otrzymuje bardzo silny sygnał
pochodzący z dolnej części obrazka (jest tam aż $625$ neuronów, które zawsze mają tę samą wartość). W przypadku tak silnego sygnału
z dolnej części obrazka neuron staje się niewrażliwy na stosunkowo słaby sygnał z górnej części obrazka.
W przypadku uczenia Oja, wagi są zmniejszane, sygnał jest korygowany, co pozwala na lepsze rozpoznanie wzorców zaszumionych.

\subsection{Eksperyment 5: duże obrazy}
W tym eksperymencie sprawdziliśmy, jak sieć Hopfielda radzi sobie z dużymi obrazami.
W tym celu wzięliśmy dwa obrazy kotów:
\begin{center}
    \includegraphics[width=0.7\textwidth]{./../../data/hopfield/photos/cat0.png}
    \includegraphics[width=0.7\textwidth]{./../../data/hopfield/photos/cat1.png}
\end{center}
A następnie zamieniliśmy je na obrazy binarne:
\begin{center}
    \includegraphics[width=0.7\textwidth]{./../../data/hopfield/eksperyment6/train/p1.png}
    \includegraphics[width=0.7\textwidth]{./../../data/hopfield/eksperyment6/train/p2.png}
\end{center}
Na których nastepnie wyszkoliliśmy sieć Hopfielda metodą Hebbowską. Ze względu na dużą liczbę neuronów
nie przeprowadzaliśmy uczenia Oja, które byłoby zbyt kosztowne obliczeniowo.
Niestety, sieć nie była w stanie rozpoznać zaszumionych obrazów kotów.
Po zaszumieniu obu obrazów na $10\%$ sieć w obu przypadkach zbiegła do tego samego minimum lokalnego:
\begin{center}
    \includegraphics[width=0.7\textwidth]{./../../data/hopfield/eksperyment6/hebb/h1.png}
\end{center}

\subsection{Eksperyment 6: Odzywskiwanie obrazu wyłącznie z szumu}
Celem tego eksperymentu było sprawdzenie, jak sieć Hopfielda poradzi sobie z odzyskaniem obrazu z samego szumu.
W tym celu przeprowadzilismy uczenie na zbiorze $animals-14x9$, zarówno metodą Hebbowską jak i Oja.
Następnie podaliśmy sieci obrazy, na których znajdował się wyłącznie szum:
\begin{center}
    \includegraphics[width=0.7\textwidth]{./../../data/hopfield/eksperyment7a/noise/n1.png}
    \includegraphics[width=0.7\textwidth]{./../../data/hopfield/eksperyment7a/noise/n2.png}
\end{center}
Wygenerowaliśmy $6$ takich obrazów. Zarówno w przypadku uczenia metodą Hebbowską jak i Oja,
 w trzech przypadkach sieć zbiegła do następującego minimum lokalnego:
\begin{center}
    \includegraphics[width=0.7\textwidth]{./../../data/hopfield/eksperyment7a/hebb/h1.png}
\end{center}
Zaś w pozostałych trzech przypadkach sieć zbiegła do następującego minimum lokalnego:
\begin{center}
    \includegraphics[width=0.7\textwidth]{./../../data/hopfield/eksperyment7a/hebb/h5.png}
\end{center}
Widzimy zatem, że pomimo, że sieć nie była w stanie odzyskać obrazu z samego szumu, to zbiegła do 
w zasadzie tego samego minimum lokalnego, jednakże z przeciwnym znakiem.
Co również ciekawe, minimum to wygląda jak połączenie wzorców uczących.
Taki sam eksperyment przeprowadziliśmy dla zbioru $large-25x25$, uzyskując następujące wyniki przy uczeniu Hebbowskim:
\begin{center}
    \includegraphics[width=0.7\textwidth]{./../../data/hopfield/eksperyment7b/hebb/h1.png}
    \includegraphics[width=0.7\textwidth]{./../../data/hopfield/eksperyment7b/hebb/h2.png}
    \includegraphics[width=0.7\textwidth]{./../../data/hopfield/eksperyment7b/hebb/h3.png}
    \includegraphics[width=0.7\textwidth]{./../../data/hopfield/eksperyment7b/hebb/h4.png}
    \includegraphics[width=0.7\textwidth]{./../../data/hopfield/eksperyment7b/hebb/h5.png}
    \includegraphics[width=0.7\textwidth]{./../../data/hopfield/eksperyment7b/hebb/h6.png}
\end{center}
Zaś w przypadku uczenia metodą Oja:
\begin{center}
    \includegraphics[width=0.7\textwidth]{./../../data/hopfield/eksperyment7b/oja/o1.png}
    \includegraphics[width=0.7\textwidth]{./../../data/hopfield/eksperyment7b/oja/o2.png}
    \includegraphics[width=0.7\textwidth]{./../../data/hopfield/eksperyment7b/oja/o3.png}
    \includegraphics[width=0.7\textwidth]{./../../data/hopfield/eksperyment7b/oja/o4.png}
    \includegraphics[width=0.7\textwidth]{./../../data/hopfield/eksperyment7b/oja/o5.png}
    \includegraphics[width=0.7\textwidth]{./../../data/hopfield/eksperyment7b/oja/o6.png}
\end{center}
Widzimy zatem, że uczenie Oja pozwoliło na lepsze odzyskanie obrazu z samego szumu,
w porównaniu do uczenia Hebbowskiego. 

\subsection{Eksperyment 7: Zapisanie możliwie wielu wzorców}
W tym eksperymencie sprawdizliśmy, jak wiele wzorców jesteśmy w stani zapisać na sieci Hopfielda o
$25$ nauronach (obrazki rozmairy $5$ na $5$). Ponieważ sieci Hopfielda są w stanie zapamiętywać 
wzorce lepiej, kiedy są one niezależne, generowaliśmy próbki losowych obrazków, a następnie sprawdzaliśmy,
czy sieć jest w stani się ich nauczyć (tzn czy po podaniu tego samego obrazu, co obraz uczący, sieć w nim pozostawała).
W ten sposób udało nam się zapisać aż $5$ wzorców, które prezentujemy poniżej:
\begin{center}
    \includegraphics[width=0.7\textwidth]{./../../data/hopfield/eksperyment8/train/p1.png}
    \includegraphics[width=0.7\textwidth]{./../../data/hopfield/eksperyment8/train/p2.png}
    \includegraphics[width=0.7\textwidth]{./../../data/hopfield/eksperyment8/train/p3.png}
    \includegraphics[width=0.7\textwidth]{./../../data/hopfield/eksperyment8/train/p4.png}
    \includegraphics[width=0.7\textwidth]{./../../data/hopfield/eksperyment8/train/p5.png}
\end{center}

\subsection{Inne eksperymenty}
W ramach projektu przeprowadziliśmy również wiele innych eksperymentów, które nie zostały tu szczegółowo opisane.
Najważniejszym z nich była nieudana próba znalezienia oscylatora w sieci Hopfielda.
Był to również jedyny eksperyment, w którym testowaliśmy sieć Hopfielda w trybie synchronicznym.

\section{Podsumowanie}
W ramach projektu zaimplementowaliśmy sieć Hopfielda, która potrafiła rozpoznawać wzorce.
Przeprowadziliśmy wiele eksperymentów, które pozwoliły nam zrozumieć, jak działa sieć Hopfielda.
W większości eksperymentów porównywaliśmy uczenie Hebbowskie z uczeniem Oja.
Wyniki eksperymentów pokazały, że uczenie Oja jest znacznie bardziej skuteczne niż uczenie Hebbowskie.
Eksperymenty przeprowadzaliśmy na różnych zbiorach danych, z reguły nie większych niż $25$ na $25$ pikseli,
jednakże w jednym eksperymencie sprawdziliśmy, jak sieć radzi sobie z dużymi obrazami.



\begin{thebibliography}{9}

    \bibitem{ref1}
    Matthias Lowe
    \textit{ON THE STORAGE CAPACITY OF HOPFIELD
    MODELS WITH CORRELATED PATTERNS},
    The Annals of Applied Probability, 1998
\end{thebibliography}
\end{document}
